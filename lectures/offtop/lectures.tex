\documentclass[12pt, a4paper]{report}

\usepackage[T2A]{fontenc}
\usepackage[utf8]{inputenc}
\usepackage[russian]{babel}
\usepackage{fancyhdr}
\usepackage{fullpage}

\addto\captionsrussian{\renewcommand{\chaptername}{Лекция}}
\addto\captionsrussian{\renewcommand{\contentsname}{Содержание}}

\title{Лекции lite}
\author{}
\parindent=0cm

\begin{document}

\maketitle

\tableofcontents
 
\chapter[7 февраля 2019]{}

\section{Введение}
\begin{itemize}
    \item \textbf{Структура лекций:}

    Будет 2 курса: классический и более современный. Мы не будем писать, к примеру, свои реализации ядер, а будем писать софт под ОС.

    \item \textbf{Структура практик:}

    Bash, последняя лаба под Windows, т.п.
\end{itemize}

\section{Эволюция ОС (этапы)}
Когда появляются операционные системы?

\subsection{Диспетчеры}
\begin{itemize}
    \item Архитектура фон Неймана (\textbf{RAM, input, output, storage}), данные и код никак не разделены в памяти

    \item Разумная идея - создать в \textbf{RAM} область памяти часто используемого кода (подпрограммы). Появилась \textbf{линковка} (вызов подпрограммы и возврат в код). Осуществлялась посредством набора конкретных указателей.

    \item Идея буфера (разделение больших объемов данных на блоки) и параллельности.

    \item \textbf{SPOOL (-ing)} : появляется контроллер связанный с \textbf{RAM, CPU, storage}.

    \item \textbf{interrupt (INT)} - сигнал контроллера к \textbf{CPU}, заставляющий его вызвать обработчик прерывания.

    \item Раньше все было ограничено физической оперативной памятью. Проблема с большим кодом, откачка, перевычисление адресов. Теперь мы не ограничены виртуальной памятью.

    \item Пакеты (модификация \textbf{storage}).
\end{itemize}

\subsection{Мультипрограммы}
\begin{itemize}
    \item \textbf{N.B.} Всегда псевдо-параллельность.

    \item Концепция разделения времени. 
        
          Передача управления другому процессу, добавление таймера, умеющего создавать прерывания.

    \item Сохранение контекста (состояние регистров) + атомарность.

    \item Проблема с разделением памяти между процессами (раньше однопрограммность подразумевала начало с одного и того же места в памяти).

    \item Появление концепции виртуальной памяти (виртуальные адреса отсчитываются от нуля).

    \item Проблема: запрос "не своей" памяти. Появление привилегированного режима. Прерывание SYSVIOLATION.

    \item Концепция библиотек (хранящихся в пакетах?) : не хотим хранить в \textbf{RAM}, а подгружать из \textbf{storage} когда нужно.

    \item Регуляция доступа к библиотекам (так появилось имя файла?, карта размещения, таблицы соответствия --- control lists )

    \item Общая концепция : виртуальная машина (некоторый процесс оказывается изолированным, абстрагированным). С позиции процесса он существует один (но хотя он "не существует", когда работают другие процессы). Из-за этого костыли, потому что процессы не знают друг о друге.

    \item 1963 - \textbf{MCP (Main Control Program)} - назвали операционной системой. Мультипрограммирование, мультиядерность ?, концепции виртуальной памяти и виртуальной машины.
\end{itemize}

\subsection{Сети}
\begin{itemize}
    \item Кругом суперкомпьютеры, продажа машинного времени. 1 час --- 120\$.

    \item Способ модуляции данных, аналогово-цифровое преобразование и обратно, передача данных.

    \item Появление терминалов. Концепция: 
        \begin{itemize}
            \item \textbf{Идентификация} --- получил карту.
            \item \textbf{Аутентификация} --- вставил карту и система узнала.
            \item \textbf{Авторизация} --- предоставление доступа.
        \end{itemize}

    \item Появление компаний, специализирующихся на расчетах.

    \item Модем ?
\end{itemize}

\subsection{Мобильные ОС}
\begin{itemize}
    \item До этого каждая ОС создавалась под конкретный компьютер. Быть программистом было непросто, так как очень часто появлялись новые компьютеры, под которые нужно было почти с нуля писать ОС.

    \item \textbf{UNICS, Bell Laboratories, MULTIX?}.

    \item Курица и яйцо : что было раньше : ОС или высокоуровневые языки?

    \item Томпсон, Керниган, Ричи последовательными шагами решают эту проблему. 
          \begin{itemize}
              \item 1 edition - assembler. Создают язык B. Потом переписывается ядро на B. 
              \item 2 edition - B. Создают Си. Создают компилятор под Си. Переписывают ядро на Си. Теперь имеют универсальную ОС. 
              \item Последняя редакция - седьмая. 
              \item Из-за разногласий, \textbf{UNICS} $\to$ \textbf{UNIX}.
          \end{itemize}
    \item История \textbf{BSD} (Университет Беркли взял исходный код \textbf{UNIX} и разработал ОС).

    \item Монтекки и Капуллети - Университеты Беркли и Стэнфорд

    \item Стэнфорд создает \textbf{SUN} в противовес. \textbf{Solaris}.

    \item \textbf{UNIX} коммерциализируется

    \item \textbf{NextStep}(startup from \textbf{BSD}) $\to$ \textbf{Darvin?} $\to$ \textbf{MacOS}.

    \item Проприетарные ОС - тормозят развитие.

    \item Ричард Столлман - свободный код. Если ты используешь свободный код, то твой код тоже должен быть свободным. \textbf{GCC}.

    \item Танненбаум - \textbf{MINIX}. Торвальдс (аспирант) высказывается за модульность, споры в сети, Линус "на слабо" пишет ядро.

    \item Линус стал участвовать в проекте \textbf{GNU/Linux?} и потребовал чтобы его имя было частью имени проекта и чтобы он "maintain - ил" ядро.
\end{itemize}

\end{document}
