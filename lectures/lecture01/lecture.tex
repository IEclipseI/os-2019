\documentclass[../lectures.tex]{subfiles}

\begin{document}

\chapter{Введение}

\section{Преподаватель}
Банщиков Дмитрий Игоревич \\
me@ubique.spb.ru

\section{Операционные системы}
Операционная система это уровень абстракции между пользователем и машиной.
Цель курса в том, чтобы объяснить что происходит в системе от нажатия кнопки в
браузере до получения результата.

Курс будет посвящен Линуксу, потому что иначе говорить особо не о чем. Линукс
это операционная система общего назначения, для машин от самых маленьких почти
без ресурсов до мощнийших серверов. Простой ответ почему линукс настолько
популярен нежели виндоус в некоторых случаях - он бесплатный.

Почеум полезно разрушить абстракцию черного ящика? Чтобы писать более
оптимизированный и функци ональный код. Иногда встр ечаются проблемы которые не
могут быть решены без знания внутренней работы ОС.

\section{Ядро и прочее}
\image{kernel-scheme.png}
Linux kernel монолитное, это оправдано для ядра, но уязвимость одной части ядра
ставит в угрозу все остальные части.
Микроядерные ОС - альтернатива монолитным (мы не будем их изучать), но с ними
сложно работать, потому что протоколы общения между частями требуют ресурсов.

UNIX-like системы - это системы предоставляющие похожий на UNIX интерфейс.

\end{document}
